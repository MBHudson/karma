\documentclass[10pt,twocolumn]{article}
\usepackage[dvips]{graphicx}
\usepackage{fancyheadings}
\usepackage{amsmath}
\usepackage{fullpage}

\begin{document}

%%%%%%%%%%%%%%%%%%%%%%%%%%%%%%%%%%%%%%%%%%%%%%%%%%%%%%%%%%%%%%%%%%%%%%%%%
%%  Authors -- please include the title of your paper and authors' names
%%  here:

\title{Attacking Automatic Wireless Network Selection}
\author{
  Dino A. Dai Zovi,
%%   \thanks{Dino. A. Dai Zovi is an Information Security Analyst working
%%     in the Attack and Exploitation Team at Bloomberg, L.P., New York,
%%     NY, 10022 and is an alumnus of the Information Design Assurance Red
%%     Team at Sandia National Laboratories in Albuquerque, NM.}
  Shane A. Macaulay \\
  ddz@theta44.org, ktwo@ktwo.ca
%%   \thanks{Shane. A. Macaulay: is an Information Security Analyst working
%%     in the Attack and Exploitation Team at Bloomberg L.P., New York, NY. 10022
%%     and is an alumnus of The Honeynet Project and other online
%%     security associations.}
}  
\date{March 20, 2005}

%%######################################################################
%%
%%  PLEASE DO NOT MODIFY ANYTHING BETWEEN THESE ## MARKS, THE PROCEEDINGS
%%  EDITORIAL STAFF WILL INCLUDE THE CORRECT PAGE NUMBERS:

%\setcounter{page}{30}

%\pagestyle{fancy}
%\setlength{\headrulewidth}{0pt}
%\lhead{}

%\cfoot{}
%\rfoot{\thepage}

%\lhead{}
%\rhead{}

\maketitle
%\thispagestyle{fancy}
%%
%%######################################################################


%%%%%%%%%%%%%%%%%%%%%%%%%%%%%%%%%%%%%%%%%%%%%%%%%%%%%%%%%%%%%%%%%%%%%%%%%
%%  Authors -- please include the abstract of your paper here:


\begin{abstract}

  Wireless 802.11 networking is becoming so prevalent that many users
  have become accustomed to having available wireless networks in
  their workplace, home, and many public places such as airports and
  coffee shops.  Modern client operating systems implement automatic
  wireless network discovery and known network identification to
  facilitate wireless networking for the end-user.  In order to
  implement known network discovery, client operating systems remember
  past wireless networks that have been joined and automatically look
  for these networks (referred to as {\it Preferred} or {\it Trusted}
  Networks) whenever the wireless network adapter is enabled.  By
  examining these implementations in detail, we have discovered
  previously undisclosed vulnerabilities in the implementation of
  these algorithms under the two most prevalent client operating
  systems, Windows XP and MacOS X.  With custom base station software,
  an attacker may cause clients within wireless radio range to
  associate to the attacker's wireless network without user
  interaction or notification.  This will occur even if the user has
  never connected to a wireless network before or they have an empty
  Preferred/Trusted Networks List.  We describe these vulnerabilities
  as well as their implementation and impact.
  
\end{abstract}

%%%%%%%%%%%%%%%%%%%%%%%%%%%%%%%%%%%%%%%%%%%%%%%%%%%%%%%%%%%%%%%%%%%%%%%%%
%%  Authors -- please include the main portion of your paper here:

\section{Introduction}

IEEE 802.11 wireless networking has demonstrated explosive growth and
popularity, especially in dense urban areas.  This has resulted in
commercial offerings of public access wireless networks ({\it
  hotspots}) in many airports, hotels, coffee shops, and even some
parks.  Large hotspot providers include T-Mobile and Verizon.  There
are even community-based projects to provide free hotspots in
community areas like Manhattan parks \cite{nycwireless}.

The prevalence of these hotspots has had an unanticipated effect on
the mechanisms in client operating systems for selecting wireless
networks.  It has been a known problem that an attacker can provide a
rogue access point with a common name (such as the default SSID of a
popular home-office access point, such as {\it linksys}).  If a nearby
wireless client has associated to a similarly-named access point in
the past, they may mistake the rogue access point for their trusted
access point.  The prescribed solution to this is to ensure that all
networks connected to are encrypted.  While this is possible when the
only networks connected to are at the home or workplace, the use of
hotspots (which must be unencrypted to provide public access) means
that users are more likely to have connected to unencrypted networks
in the past.

According to a Gartner research report (quoted in
\cite{lohr04desktop}), Microsoft Windows and Apple MacOS are the two
most prevalent desktop operating systems, with 96\% and 2.8\% market
share, respectively.  Our research examined the latest releases of
these desktop operating system families, Windows XP Service Pack 2 and
MacOS X 10.3.8.  We describe in detail the mechanisms used by the
latest releases of these operating systems for automatic network
selection and how they may be attacked.  Our research in this area has
also uncovered vulnerabilities in the implementation of these
algorithms.  Namely, the implementations set the wireless network
adapter in a ``parked'' mode when the user is not associated to a
network.  In this mode, the card's ``desired SSID'' setting is set to
a value that the implementor has expected to never exist as an
available network.  When a network by this name does exist (or at
least appears to), however, the wireless network card will
automatically associate.  This occurs without user interaction or
notification.

This paper is organized as follows.  Section \ref{background} provides
an overview of the concepts involved in 802.11 wireless networking and
describes related work in 802.11 client security research.  Section
\ref{wlanselection} documents in detail how Windows XP and MacOS X
implement automatic wireless network selection.  Section
\ref{attacking} describes vulnerabilities in the implementations of
automatic wireless network selection and weaknesses in the algorithms
they use.  In section \ref{driver}, we describe the implementation of
a customized software access point driver to exploit the previously
described vulnerabilities.  Section \ref{conclusion} discusses the
ramifications of these vulnerabilities and describes future work in
this area of research.

\section{Background}
\label{background}

The IEEE 802.11 standard \cite{ieee99wireless} specifies medium access
control (MAC) and physical layer (PHY) operation for local area
wireless networking.  It is now the most common standard for wireless
networking and most laptops now include integrated wireless network
cards using the 802.11 standard.

The 802.11 standard defines two entities in a wireless network, the
Station (STA) and Access Point (AP).  A wireless network, or Basic
Service Set (BSS) as it is referred to in the standard, may be created
in two configurations: the Independent Basic Service Set (IBSS) and
the Extended Service Set (ESS).  Every IBSS or ESS is named by a
Service Set Identifier (SSID), usually a 7-bit ASCII string with a
length not exceeding 32 characters.  An IBSS, or {\it Ad-Hoc network},
is created by any number of stations without requiring any Access
Points.  The more common network configuration, an ESS or {\it
Infrastructure network}, is created with one or more Access Points
creating a Distribution System (DS) that clients may join.  The most
common configuration is an Infrastructure network with one Access
Point.

In order to understand the attacks introduced in this paper, we must
detail the 802.11 frame types used for locating and joining networks.
In order to announce its presence, each AP in an ESS broadcasts {\it
Beacon} frames containing the SSID and various characteristics of the
ESS, including supported data rates as well as whether encryption is
enabled.  Stations may locate nearby networks by observing these
Beacon frames or by sending {\it Probe Request} frames.  The Probe
Request frame contains the SSID the STA is looking for as well as the
transfer rates supported by the STA.  The SSID may be an empty string
indicating that the frame is a broadcast Probe Request.  Access Points
within signal range typically respond to both broadcast Probe Requests
and Probe Requests containing their SSID with a {\it Probe Response}
frame containing the network's SSID\footnotemark, supported rates, and
whether the network is encrypted.  If encryption is enabled, the STA
must authenticate prior to associating to the network.  This is
performed through a sequence of {\it Authentication} frames.  If the
STA has properly authenticated itself or the network does not require
authentication, the STA will send an {\it Association Request} frame
to which the AP responds with an {\it Association Response} frame.  At
this point the STA may begin to participate on the wireless network.

\footnotetext{
  In response to a well-publicized practice called {\it Wardriving}
  \cite{shipley01wardriving}
  % where a curious user attempts to locate as
  %many Access Points as possible, 
  many Access Points support an
  option to ignore broadcast Probe Requests and
  not broadcast the SSID in Beacon frames.  Such access points are
  referred to as a {\it closed} or {\it hidden}.
  }

It is important to point out that while the standard specifies how a
STA joins a ESS, how the ESS is chosen is unspecified.  The
specification allows for roaming between base stations with the same
SSID, but there is no mention of whether the base station should be
authenticated or simply trusted.  The specification also does not
address how a wireless client is to select an available network, and
this has been left up to implementation by hardware and operating
system software vendors.

Although the security of wireless networks has been the subject of
much research, the security of wireless clients has not had the same
level of focus.  There has been some level of research and related
work that we have been able to build upon which we will briefly
describe.

Vulnerabilities in the 802.11 MAC layer have been discovered that
allow nearby attackers to launch a denial-of-service attack against
nearby wireless clients, effectively ``jamming'' the wireless
network \cite{nobles05wlandos}.

The Wireless LAN 802.11b Security FAQ \cite{klaus02wlan} describes
several attacks against wireless clients.  The FAQ mentions how a
cloned base station ({\it Evil Twin}) can produce a stronger signal
than the legitimate base station and divert unsuspecting clients away
from the original base station.  The FAQ also mentions that nearby
wireless attacker can target vulnerable TCP/IP services or perform
denial-of-service attacks against a nearby wireless client.

Max Moser's Hotspotter \cite{moser04hotspotter} is an
automated wireless client penetration tool for Linux.  Hotspotter
places the attacker's wireless network card in monitor mode to
passively listen for Probe Request frames.  For each Probe Request
frame received, the requested SSID is compared to a list of known
hotspot names.  If there is a match, the wireless network card is
reconfigured to act as an access point with that name.

\section{Wireless Network Selection}
\label{wlanselection}
\subsection{Microsoft Windows XP Wireless Auto Configuration}

\begin{figure}
\begin{verbatim}
Begin:  
State = Unconnected
// Build list of visible networks (ANL)
AvailableNetworks = ScanForAvailableNetworks()

// Step through PNL in order until a network
// from the ANL is found and connected to
foreach n in PreferredNetworks
  if AvailableNetworks contains n
  then ConnectToWirelessNetwork(n)
  if State == Connected then return

// If unable to connect to any networks in the
// intersection of the PNL and ANL, check for
// closed networks by stepping through PNL in
// order and attempting each network explicitly
foreach n in PreferredNetworks
  ConnectToWirelessNetwork(n)
  if State == Connected then return

// If unable to connect to any network in the
// PNL and the PNL contains an Ad-Hoc network,
// configure card for the first Ad-Hoc network
// in the PNL.  Otherwise, if the configuration
// setting "Connect to Non-preferred Networks"
// is enabled, step through ANL in order and
// attempt to connect to each one.
if PreferredNetworks contains an Ad-Hoc network
  then ConfigureAdHocNetwork
else
  if ConnectToNonPreferredNetworks == True
  then foreach n in AvailableNetorks
         ConnectToWirelessNetwork(n)
         if State == Connected then return

  // If not connected thus far, generate a 
  // random SSID, wait, and restart algorithm
  SetSSID(GenerateRandomSSID())
  SleepForOneMinute()
  Goto Begin
\end{verbatim}
\caption{Pseudocode for Wireless Auto Configuration Algorithm}
\label{wzcalg}
\end{figure}

Microsoft Windows XP and Windows Server 2003 were the first Microsoft
operating systems to fully support 802.11 wireless networking on the
client and server, respectively.  Under these operating systems,
wireless network configuration and detection is performed through a
process called Wireless Auto Configuration.

Central to the configuration and operation of Wireless Auto
Configuration are the {\it Preferred Networks List} (PNL) and {\it
Available Networks List} (ANL).  The PNL is an ordered list of the
networks that the user has connected to in the past.  The ANL is an
ordered list of all the Access Points that responded to a broadcast
Probe Request in the last wireless network scan.  Whenever the user
connects to a new network, that network's name (SSID) is added to the
head of the PNL.  Windows provides user interfaces for viewing the
Available Networks List, managing the Preferred Networks List, and
configuring the behavior of the Wireless Auto Configuration algorithm,
described below.

The Windows XP and Server 2003 Wireless Auto Configuration algorithm
\cite{cableguy02wac} is presented in a pseudocode notation in Figure
\ref{wzcalg} but will also be presented in narrative and a network
trace in order to fully illustrate its behavior.

% WZC Alg in words

The algorithm begins by building the Available Networks List by
sending a scan request to the wireless network card.  This results in
a broadcast 802.11 Probe Request with an empty SSID being sent over
every channel.  Probe Responses are collected by the card in the order
they are received and this list of networks along with configuration
details such as encryption status and available rates is returned to
the operating system.  The algorithm then traverses the Preferred
Networks List in order and if a preferred network is found in the
Available Networks List, Windows attempts to connect to the wireless
network.  A connection is attempted to each preferred network found in
the Available Networks List until there is a successful connection.

If no preferred networks were found in the Available Networks List or
no connection attempts were successful, Windows attempts a second pass
of the Preferred Networks List trying to connect to each network in
the Preferred Networks List in order regardless of whether the network
exists in the Available Networks List.  This second pass is performed
in case any of the networks are ``closed networks'', a common
deviation from the standard where the SSID is not placed in 802.11
Beacon frames and broadcast Probe Requests are not responded to.

If the wireless network card is not connected and there are one or
more Ad-Hoc networks in the Preferred Networks List, Wireless Auto
Configuration will configure the card for the most preferred Ad-Hoc
network available.  If none are available, Wireless Auto Configuration
will create the most preferred Ad-Hoc network and the algorithm
terminates.

If there have been no connections thus far and there are no Ad-Hoc
networks in the Preferred Networks List, the algorithm examines the
{\it Connect To Non-preferred Networks} flag.  If the flag is enabled
(it is disabled by default), Windows will attempt to connect to each
network in the Available Networks List in order.  If the flag is not
set, the network card is ``parked'' in Infrastructure mode with a
randomly generated SSID for 60 seconds at which point the algorithm
restarts.

% WZC Alg in packets/drivers

\begin{figure}
\begin{verbatim}
1) 03:11:39.266743 BSSID:ff:ff:ff:ff:ff:ff
  DA:ff:ff:ff:ff:ff:ff SA:00:02:2d:2b:a5:35
  Probe Request (^]^V^K^A^T^B^T^A^X^E^Y^V...)
    [1.0 2.0 5.5 11.0 Mbit]
2) 03:11:39.400194 BSSID:ff:ff:ff:ff:ff:ff
  DA:ff:ff:ff:ff:ff:ff SA:00:02:2d:2b:a5:35
  Probe Request () [1.0 2.0 5.5 11.0 Mbit]
3) 03:11:40.426391 BSSID:ff:ff:ff:ff:ff:ff
  DA:ff:ff:ff:ff:ff:ff SA:00:02:2d:2b:a5:35
  Probe Request (^]^V^K^A^T^B^T^A^X^E^Y^V...)
  [1.0 2.0 5.5 11.0 Mbit]
4) 03:11:42.377495 BSSID:ff:ff:ff:ff:ff:ff
  DA:ff:ff:ff:ff:ff:ff SA:00:02:2d:2b:a5:35
  Probe Request (aye) [1.0 2.0 5.5 11.0 Mbit]
5) 03:11:44.432180 BSSID:ff:ff:ff:ff:ff:ff
  DA:ff:ff:ff:ff:ff:ff SA:00:02:2d:2b:a5:35
  Probe Request (bee) [1.0 2.0 5.5 11.0 Mbit]
6) 03:11:46.485148 BSSID:ff:ff:ff:ff:ff:ff
  DA:ff:ff:ff:ff:ff:ff SA:00:02:2d:2b:a5:35
  Probe Request (sea) [1.0 2.0 5.5 11.0 Mbit]
7) 03:11:48.500975 BSSID:ff:ff:ff:ff:ff:ff
  DA:ff:ff:ff:ff:ff:ff SA:00:02:2d:2b:a5:35
  Probe Request (^D^F^R^_^]^R^^^\^G^H^J^F...)
  [1.0 2.0 5.5 11.0 Mbit]
\end{verbatim}
\caption{Wireless Auto Configuration Algorithm packet trace}
\label{wacpkts}
\end{figure}

When analyzing the algorithm from an attacker's point of view, it is
most helpful to examine it at the 802.11 protocol layer.  Figure
\ref{wacpkts} is a packet trace of a single iteration of the Wireless
Auto Configuration algorithm running on a laptop with a fresh install
of Windows XP with Service Pack 2.  The network trace was recorded on
a nearby laptop with its wireless network card in ``monitor mode'', a
special operating mode where all received 802.11 frames are returned
to the operating system.  The output has been edited slightly for
cleaner presentation by numbering frames, removing duplicate frames,
wrapping lines cleanly, and abbreviating long random SSIDs.

Frames 1-3 show the first phase of the algorithm where the network
card is ``parked'' with a random SSID, but sends a broadcast Probe
Request frame when Wireless Auto Configuration initiates a scan
request.  Frames 4-6 result from the second phase of the algorithm
where a connection to each of the networks in the Preferred Networks
List is attempted in order.  We observe that the networks in the
Preferred Networks List are (in order): ``aye'', ``bee'', and
``sea''.  Since none of these networks were found, Frame 7 shows the
wireless card being parked with another random SSID.  In the full
trace, 16 Probe Requests for the last random SSID were observed over
one minute before the algorithm on the observed wireless client
restarted.

\subsection{Apple MacOS X AirPort}

Apple's MacOS X operating system supports wireless networking with
Apple's AirPort and AirPort Extreme 802.11 wireless networking
products for 802.11b and 802.11g, respectively.

MacOS X allows the user to select three wireless network selection
modes: always connecting to a specific wireless network, connecting to
the most recently associated network or automatically connecting to
the unencrypted network with the strongest signal.  The user may
also manually select an alternate available network or enter the name
of a closed network.  In December 2003, a vulnerability was published
whereby if an attacker can get a MacOS X user to join their wireless
network, the attacker's DHCP server may provide an DHCP option
specifying a directory server that the user's machine will use as an
authentication server.  MacOS X 10.3.3 addressed this vulnerability by
modifying the system to maintain a list of trusted wireless networks.

The MacOS X trusted wireless networks is a system-wide list of
networks the user has connected to and opted to add to the list.  No
user interface is provided to view or modify this list.  The list, in
fact, is even fairly difficult to find.  It is stored as an XML file
which is base-64 encoded and stored within another XML file containing
basic wireless network adapter settings.  Cursory Internet searches
reveal that the existence of this list is completely undocumented, yet
many have discovered that they can delete the entire file to clear the
list of trusted wireless networks.

MacOS X begins the search for trusted wireless networks when a user
logs in or the machine awakes from sleep.  The search begins with the
network the machine was most recently associated with.  If this
network is not found, each network in the trusted network is attempted
in order.  If none of these networks are found, a dialog is presented
to the user stating that none of their trusted wireless networks could
be found and asking whether they would like to join the unencrypted
network with the strongest signal and optionally remember the network
as a trusted wireless network.  If the user opts not to connect to the
selected network, the wireless network card is ``parked'' awaiting
user interaction.  In this state, the wireless network card remains in
Infrastructure mode, however, it is assigned a either a constant
``dummy'' SSID or a dynamic SSID that is chosen when the driver is
initialized (system boot or resume).  Both settings have been seen,
although the dummy SSID appears to be set at system boot and when
awakening from sleep, but the dynamic SSID is only set when the user
logs in.  Broadcast Probe Requests are sent on each channel roughly
every two minutes or when the available networks need to be presented
to the user.

\section{Attacking Wireless Network Selection}
\label{attacking}

Through the detailed examination and documentation of the processes used
by the two most prevalent desktop operating systems, a number of
vulnerabilities in these processes were uncovered.  The specific
vulnerabilities in each implementation is detailed below and the
attacks are summarized in terms of which configurations are vulnerable
in section \ref{attacksummary}.

\subsection{Microsoft Windows Wireless Auto Configuration}

Windows' Wireless Auto Configuration has several serious weaknesses:
all networks and their precedence in the Preferred Networks List are
revealed, ad-hoc networks are automatically created, and ``parking''
with a random SSID does not prevent associating to an Access Point
with custom firmware modifications.  Moreover, this association may
occur without user interaction or notification.  Through these
weaknesses, the wireless client is most vulnerable when it is not
associated to a surrounding network.

% Preferred Networks List is revealed
When no preferred networks are discovered in the Available Networks
List built in the first phase of the Wireless Auto Configuration
algorithm, the algorithm attempts to connect to each network in the
Preferred Networks List in order.  An attacker within signal range
(potentially assisted by high-gain antennae and signal boosters) may
passively monitor a single wireless channel and observe the Probe
Requests for each network connection attempted.  The attacker may use
this information to recreate the victim's Preferred Networks List.
However, only the names and precedence of the networks are revealed,
their encryption status is not.  With this knowledge, the attacker may
create a software Access Point with the SSID of one of the networks in
the victim's Preferred Network List.  If the client is expecting the
network to be encrypted, the connection will fail and the attacker may
simply attempt the next network in their recreated copy of the
victim's Preferred Networks List.  If {\it any} of the networks in the
Preferred Networks List are not encrypted, the attacker will have an
opportunity to create a look-alike network that the client will join.

% Ad-Hoc Networks are automatically created

If there are any Ad-Hoc networks in the Preferred Networks List and
none are found in the Available Networks List, the client will become
the first node in the Ad-Hoc network.  If this network is not
configured to be encrypted, any other client within wireless signal
range may join this network.  If the network is configured with WEP
encryption, any of the known WEP attacks may be performed against it.
Several of these attacks are facilitated by the fact that the network
interface is configured using Windows' Automatic Private IP Addressing
whereby the interface is given an automatically selected IP address
from the link local IP address space 169.254.0.0/16 documented in RFC
3330 \cite{rfc3330}.  Because the interface is configured, Windows
will periodically send NetBIOS broadcasts, continually supplying the
attacker with WEP encrypted data packets.  Once the network is joined
by the attacker, they can proceed to discover the self-assigned IP
address used by the victim's wireless network interface.  This may be
done by sniffing the network for the previously mentioned NetBIOS
broadcasts or brute-forcing the link local IP address space with
Address Resolution Protocol (ARP) requests for each possible IP
address.

% Random SSID

As described above, when the Wireless Auto Configuration algorithm
sleeps for 60 seconds before restarting, the wireless network card is
``parked'' by placing the card in Infrastructure mode with a random
SSID.  Other card settings such as authentication mode or encryption
are not changed.  When the card is placed in this state, it assumes
its normal behavior attempting to connect to a network with that name.
If there are no entries in the Preferred Networks List or the last
entry is an unencrypted network, the network adapter will attempt to
connect to an unencrypted network with this random SSID.  If an
attacker can provide a network with that name, the wireless client
will automatically associate to the attacker's wireless network.

This attack may be performed by modifying the firmware of an Access
Point to respond with Probe Responses to Probe Requests for {\it any
SSID}.  This attack is facilitated by using the newer firmware-less
wireless network cards (such as those using Atheros chipsets).  We
describe the necessary driver modifications and implementation of this
attack in Section \ref{driver}.  As this network is joined without the
Wireless Auto Configuration Service's knowledge, the user interface
does not inform the user that they have associated to a network.  In
this case, the client may associate, receive an IP address from a DHCP
server, and become reachable on the network all while the interface
informs the user that they are not currently connected to a network.

\subsection{MacOS X AirPort}

% When parked, card is given random SSID w/ WEP enabled, but WEP key
% is static (doh!).  Looks for ``shared key'' authentication w/ wep
% key 0102030405.   Only AirPort classic though, APX is perfect :(.

MacOS X's AirPort implementation shares several of the pitfalls
identified in the Windows Wireless Auto Configuration implementation
but avoids others.  Similar to the vulnerabilities described above,
MacOS X AirPort reveals the list of trusted wireless networks and
machines using the 802.11b AirPort hardware may associate to a
specially-configured access point without user interaction or
notification.

Like Microsoft Windows' Wireless Auto Configuration, MacOS X AirPort
reveals the list of trusted wireless networks when the system is
looking for a network to associate to.  This is, however, an important
difference because MacOS X AirPort does not continually look for the
user's trusted wireless networks, only when a user logs in or the
system resumes from sleep.  This makes it more difficult for an
attacker to cause the user to join their rogue access point as it adds
temporal constraints, requiring the attacker to be in the right place
at the right time.
% XXX: What about when the user is dissassociated from the current
% network via spoofed disassoc frames?

The drivers for the older 802.11b AirPort hardware, like Wireless Auto
Configuration, also keep the wireless card up in a ``parked'' state
when not actively associated.  As described above, the card's desired
SSID is set to either a dynamic value or a static ``dummy SSID''.  In
order to further prevent unintended associates, the card is set with
WEP enabled, requiring a rogue Access Point to know this WEP key in
order to cause the client to join automatically.  This WEP key,
however, is the hard-coded, static 40-bit key (in hexadecimal)
0x0102030405.  Our custom Access Point software configured with this
WEP key in Shared Key authentication mode successfully causes nearby
AirPort cards to automatically associate without requiring any user
interaction or providing any user notification.  The AirPort menu
applet will, however, light up, and if the user clicks on it, it will
indicate that the user is connected to a network.  It should be noted
that the newer AirPort Extreme hardware and drivers do not leave the
card in a ``parked'' state vulnerable to this attack.  When the card is
not associated no wireless traffic is sent unless the user requests a
scan for visible networks.

\subsection{Summary of Attacks}
\label{attacksummary}

The attacks described above focus on the ability of the attacker to
provide a network that the wireless client will automatically join.
The attacker may perform this by either discovering the networks that
the client prefers or by using a special software access point that
masquerades as any SSID.  We describe the construction of such a
software AP in section \ref{driver}.  Either way, the attacker must
make this network present when the victim is looking for a wireless
network to join.

If the victim is running MacOS X, the attacker must be present when
the user logs in or the system wakes from sleep.  Regardless of the
operating system, however, if the user is currently associated to a
nearby network, the attacker may forcibly cause the victim to restart
the search for available networks.  The attacker may achieve this by
spoofing 802.11 Disassociation frames from the base station that the
victim is associated to.  These frames are always sent in the clear,
even if the network is encrypted, and the only information the
attacker requires to forge them is the hardware address of the base
station, which is readily available from the Beacon frames the base
station is required to continuously transmit.  If the targeted client
is currently configured in Ad-Hoc mode, there is no known way to cause
it to restart the search for a preferred network.

At this point, the attacker may learn the victim's preferred or
trusted networks as they attempt to rejoin an available network or
respond that they are the base station for every requested SSID.  If
the targeted client looks for one or more unencrypted networks, they
will automatically join the attacker's access point.

As a special case that was mentioned above, if a nearby Windows XP
client is not currently associated to a network, it may still
associate to a network with a random SSID.  When this is the case,
this connection will occur without notifying the user and the user
interface will still report that the machine is not currently
connected to any wireless networks.

The attacks detailed above have been observed and verified against a
Windows XP laptop with PCMCIA PrismII and Orinoco Hermes-based 802.11b
wireless network cards as well as against a G4 Macintosh with an
internal AirPort 802.11b wireless card.  We also tested a Windows XP
SP2 laptop with an internal 802.11a/b/g card based on the Atheros
chipset and found that while it still did send out Probe Requests for
randomly-generated SSIDs, it would not join these networks
automatically.  By testing a newer G4 Powerbook, we found that the
newer Apple AirPort Extreme 802.11b/g cards did not probe for the
dynamic SSID and dummy SSIDs described above.  We hypothesize that the
newer generation of wireless network cards that perform more of the
802.11 handling in software are more flexible and robust than their
firmware-based predecessors.

\section{Attack Implementation}
\label{driver}

\begin{figure}[t]
  \begin{verbatim}
1) 00:49:04.007115 BSSID:ff:ff:ff:ff:ff:ff
   DA:ff:ff:ff:ff:ff:ff SA:00:e0:29:91:8e:fd
   Probe Request (^J^S^V^K^U^L^R^E^H^V^U...)
   [1.0* 2.0* 5.5* 11.0* Mbit]
2) 00:49:04.008125 BSSID:00:05:4e:43:81:e8
   DA:00:e0:29:91:8e:fd SA:00:05:4e:43:81:e8
   Probe Response (^J^S^V^K^U^L^R^E^H^V^U...)
   [1.0* 2.0* 5.5 11.0 Mbit] CH: 1
3) 00:49:04.336328 BSSID:00:05:4e:43:81:e8
   DA:00:05:4e:43:81:e8 SA:00:e0:29:91:8e:fd
   Authentication (Open System)-1: Succesful
4) 00:49:04.337052 BSSID:00:05:4e:43:81:e8
   DA:00:e0:29:91:8e:fd SA:00:05:4e:43:81:e8
   Authentication (Open System)-2: 
5) 00:49:04.338102 BSSID:00:05:4e:43:81:e8
   DA:00:05:4e:43:81:e8 SA:00:e0:29:91:8e:fd
   Assoc Request (^J^S^V^K^U^L^R^E^H^V^U...)
   [1.0* 2.0* 5.5* 11.0* Mbit]
6) 00:49:04.338856 BSSID:00:05:4e:43:81:e8
   DA:00:e0:29:91:8e:fd SA:00:05:4e:43:81:e8
   Assoc Response AID(1) :: Succesful
  \end{verbatim}    
  \caption{Windows XP host associating to random SSID}
  \label{winxprandom}
\end{figure}

\begin{figure}[t]
\begin{verbatim}
1) 14:14:18.778980 BSSID:ff:ff:ff:ff:ff:ff
   DA:ff:ff:ff:ff:ff:ff SA:00:30:65:00:e9:65
   Probe Request (00-30-1e-37-7a-44-7f49c08d)
   [1.0 2.0 5.5 11.0 Mbit]
2) 14:14:18.779845 BSSID:00:05:4e:43:81:e8
   DA:00:30:65:00:e9:65 SA:00:05:4e:43:81:e8
   Probe Response (00-30-1e-37-7a-44-7f49c08d)
   [1.0* 2.0* 5.5 11.0 Mbit] CH: 1, PRIVACY
3) 14:14:18.837813 BSSID:ff:ff:ff:ff:ff:ff
   DA:ff:ff:ff:ff:ff:ff SA:00:30:65:00:e9:65
   Probe Request (00-30-1e-37-7a-44-7f49c08d)
   [1.0 2.0 5.5 11.0 Mbit]
4) 14:14:18.906312 BSSID:00:05:4e:43:81:e8
   DA:00:05:4e:43:81:e8 SA:00:30:65:00:e9:65
   Authentication (Shared Key)-1: Succesful
5) 14:14:18.907962 BSSID:00:05:4e:43:81:e8
   DA:00:30:65:00:e9:65 SA:00:05:4e:43:81:e8
   Authentication (Shared Key)-2 [Challenge] 
6) 14:14:18.909513 BSSID:00:05:4e:43:81:e8
   DA:00:05:4e:43:81:e8 SA:00:30:65:00:e9:65
   AuthenticationAuthentication (Shared-Key)-3
   Data IV:  0 Pad 0 KeyID 0
7) 14:14:18.910320 BSSID:00:05:4e:43:81:e8
   DA:00:30:65:00:e9:65 SA:00:05:4e:43:81:e8
   Authentication (Shared Key)-4: 
8) 14:14:18.911565 BSSID:00:05:4e:43:81:e8
   DA:00:05:4e:43:81:e8 SA:00:30:65:00:e9:65
   Assoc Request (00-30-1e-37-7a-44-7f49c08d)
   [1.0 2.0 5.5 11.0 Mbit]
9) 14:14:18.912575 BSSID:00:05:4e:43:81:e8
   DA:00:30:65:00:e9:65 SA:00:05:4e:43:81:e8
   Assoc Response AID(1) : PRIVACY : Succes
\end{verbatim}
\caption{AirPort client parked with dynamic SSID associating}
\label{osxapdynamic}
\end{figure}

\begin{figure}[htb]
\begin{verbatim}
1) 14:25:05.926870 BSSID:ff:ff:ff:ff:ff:ff
   DA:ff:ff:ff:ff:ff:ff SA:00:30:65:00:e9:65
   Probe Request (dummy SSID *{M-^R^LIM-^O...)
     [1.0 2.0 5.5 11.0 Mbit]
2) 14:25:05.927916 BSSID:00:05:4e:43:81:e8
   DA:00:30:65:00:e9:65 SA:00:05:4e:43:81:e8
   Probe Response (dummy SSID *{M-^R^LIM-^O...)
   [1.0* 2.0* 5.5 11.0 Mbit] CH: 1
3) 14:25:06.095809 BSSID:00:05:4e:43:81:e8
   DA:00:05:4e:43:81:e8 SA:00:30:65:00:e9:65
   Authentication (Open System)-1: Succesful
4) 14:25:06.096565 BSSID:00:05:4e:43:81:e8
   DA:00:30:65:00:e9:65 SA:00:05:4e:43:81:e8
   Authentication (Open System)-2: 
5) 14:25:06.098862 BSSID:00:05:4e:43:81:e8
   DA:00:05:4e:43:81:e8 SA:00:30:65:00:e9:65
   Assoc Request (dummy SSID *{M-^R^LIM-^O...)
   [1.0 2.0 5.5 11.0 Mbit]
6) 14:25:06.099773 BSSID:00:05:4e:43:81:e8
   DA:00:30:65:00:e9:65 SA:00:05:4e:43:81:e8
   Assoc Response AID(1) :: Succesful
\end{verbatim}  
\caption{AirPort client parked with dummy SSID associating}
\label{osxapdummy}
\end{figure}

%% Firmware limitations also enforce 802.11 frame handling. This handling
%% is tied to the operational mode of the device.  This is typically one
%% of several modes most frequently used are one of Adhoc, HostAP and
%% monitor.

Implementation of attacks against 802.11 networks and clients are
often hampered by the limitations imposed by wireless network card
firmware.  For example, certain frame types may be handled directly by
the firmware and will not be made available to the host operating
system.  This behavior prevents many commercial off-the-shelf 802.11
wireless network cards from being used as an arbitrary attack
platform.  Initial implementations of the attacks presented in this
paper were attempted using wireless network cards based on the PrismII
chipset.  These chipsets feature a {\it HostAP} operating mode
allowing the card to act as an Access Point.  Other cards based on the
Orinoco Hermes chipset, for example, require an alternate firmware to
provide this functionality.  This operating mode makes management
frames, including Authentication and Association Request frames,
available to the operating system to allow the operating system or a
running user process to provide station authentication and management
services.  Notably missing, however, are the Probe Request frames
necessary to implement our attacks.  Due to the real-time requirements
on Probe Request handling, Probe Request frames are handled internally
by the card firmware.  This limited our attack to sniffing for Probe
Requests and immediately re-configuring the card to serve as an access
point for the requested SSID, serializing the attack to targeting one
wireless client at a time.

%% In addition, the frame handling semantics of a given mode may preclude
%% useful information becoming available to the host operating system.
%% For instance, when functioning in HostAP mode firmware frame handling
%% will drop received unicast frames that are addressed to it.  There are
%% several instances in which these frames might be useful. 

%% will be dropped, so as not to be visible to the host operating system.
%% Several workaround solutions can be utilized with varying levels of
%% success.  A platform that integrates several radio's all functioning
%% in parallel can often allow for a hostile AP to be notified of any
%% other activity when supplemented with a monitor mode radio.  The
%% synchronization of between these two radios is fairly straight forward
%% but has some limitations.  Some being that when the active interface
%% is placed in a new operational mode, host relationships will be
%% expired and any ongoing relationship will be terminated.  This type of
%% platform also restricts the operator in that she must administratively
%% decide on which mode and configuration to assume, thereby eliminating
%% the possibility for an elegant headless configuration.

%% The majority of 802.11 radios primary interface has been a
%% manufacturer controlled firmware.  The firmware typically enforces
%% several aspects of the radio's functionality across several
%% operational domains.  The radio frequency and power output are
%% governed in a way dictated by its locality.  For instance, 802.11b and
%% 802.11g have an operating frequency of 2.4GHz.  This indicates a
%% channel range that includes 2.412 to 2.484GHz, for a total of 14 radio
%% channels.  North American radios are limited to channels 1-11, Europe
%% adds the ranges for 12 and 13 where Japan has the full spectrum
%% available including channel 14.

%% The recent arrival of software radios (especially the Atheros
%% chipset),

The newer generation of firmware-less wireless network cards allow
significantly more flexibility in the implementation of attacks
against IEEE 802.11.  Wireless network cards based on chipsets
manufactured by Atheros Communications, for example, do not include a
traditional firmware providing station and/or access point
functionality.  Instead, all of this functionality is handled in
software on the host and a binary-only Hardware Abstraction Layer
(HAL) module is shipped to vendors and driver authors to provide
low-level functionality.  This HAL module enforces communications
regulation compliance and provides a consistent interface to the
different implementations manufactured by Atheros.  This binary-only
HAL is also used by the open-source drivers for this family of
wireless network cards.

To implement our attacks, we have taken the open-source MADWiFi driver
for Linux \cite{leffler02madwifi} and made several modifications.  We
implemented several trivial modifications including disabling SSID
validation and rewriting the SSID field of incoming Probe Requests and
Association Request frames with the configured SSID of the software
access point.  Outgoing frames required no modification.  This allowed
us to easily fool the rest of the driver into serving any SSID
requested.  This implementation allows us to create a wireless network
that masquerades as any network requested by a client within range.

Figures \ref{winxprandom}, \ref{osxapdynamic}, and \ref{osxapdummy}
show 802.11 frame traces of our attack driver exercising the
previously detailed vulnerabilities in wireless network selection.  In
each of the three cases, the wireless client associated to a network
name that the implementation did not expect to exist but was responded
to by our attack driver.  This allowed us to cause the client to
associate to our network without any user interaction or notification.

%% The current implementation allows for the creation of an {\it
%% all-SSIDs} network that simultaneously allows lower layer monitor
%% functionality (the ability to monitor client probe request frames).
%% This dual-purpose model allows us to cross the underlying abstraction
%% layers in a way unanticipated by the client node.

%% These capabilities were enabled with a relatively small patch to the
%% overall driver code.  ESSID's are only used in a limited selection of
%% 802.11 frame types, namely probe and association requests.  This keeps
%% the modification of driver code a simple task.

Further work on the attack driver component will focus on creating
a network interface that will enable reception and transmission of raw
frames, even while serving as an access point.  This functionality
will allow a user to utilize existing tools requiring passive sniffing
capabilities while also enabling active attacks.  In addition, it may
enable true 802.11 MAC-layer man-in-the-middle attacks.

\section{Conclusion}
\label{conclusion}

We have shown that there are both architectural and implementation
vulnerabilities in common wireless network selection algorithms.  The
broadcasting of specific Probe Request frames containing the SSID of a
user's desired network allows a nearby attacker to learn the contents
and precedence of the user's desired networks.  With this knowledge,
the attacker may create a rogue access point with this SSID that the
user will automatically associate to.

In addition, both implementations also leave the wireless network
cards in a vulnerable state when they are not currently associated to
a network.  Windows XP configured the network card with a randomly
generated SSID while MacOS X AirPort uses a dynamic, but not random,
SSID.  This is done assuming that no access points will ever serve
these SSIDs.  We created a special access point that responded to any
Probe Request, allowing us to serve any SSID requested by a client
within range.  Experiments with this driver revealed serious
vulnerabilities in the previously mentioned driver behavior.  We
discovered that when our access point responded to probes for these
special SSIDs, we could cause clients to associate to our network
without any user interaction and little or no user notification.
Under Windows XP, the wireless configuration user interface will
report that the user is not associated to any wireless networks.
Network interface configuration elements, however, will report that
the interface is connected and configured with an IP address.  MacOS X
AirPort will not request permission to join a previously unknown
wireless network, but will correctly report that it is connected.

While there is a known growth of client-side attacks, there has been a
lack of knowledge regarding how they may be realistically used by an
attacker to achieve their objectives.  Frequently, client-side attacks
are described as being a danger if the attacker can cause the victim
to view a malicious web page or e-mail.  While it is evident that an
adversary may send a large number of e-mails to harvested addresses
within an organization to increase the chances of an internal user
viewing the message and possibly viewing the malicious content within,
this is in no way a stealthy attack.  Similarly, coercing internal
users to view a malicious web site requires an element of social
engineering.  We believe that a wirelessly connected attacker is in
the best position to deploy attacks against client-side applications.

We have shown that an attacker can force wireless clients within
signal range to join an attacker-controlled network.  This opens up a
very viable and dangerous avenue for passively exploiting client-side
vulnerabilities using man-in-the-middle techniques.  Drawing from
existing cyber-warfare principles (\cite{parks01cyberwar}), we are
calling this scenario {\it medium-range cyber-warfare}.  By exploiting
the vulnerabilities described in this paper, we may cause our opponent
to join a wireless network where we control the entire network
environment.  Since we control the entire network, we may leverage
this to control the opponent.  We may use this approach to attack any
wireless-enabled clients within range, and we may employ strong
wireless antennas and transmitters to increase this range.

Many attacks can be easily implemented as fake services responding to
requests by client-side applications.  Bare-bones emulation servers
can often be developed with ease \cite{provos04honeyd} or existing
packages may be modified to exercise vulnerabilities in client code.
For example, while implementing a malicious POP3 server is quite
simple due to the simplicity of the POP3 protocol, implementing a
malicious SMB server may require modifying an existing implementation
such as the open-source Samba SMB/CIFS server\cite{samba}.

%% Client attacks can then be seen as having the inherent nature of
%% having the capability to exercise all reachable code paths with no
%% requirement for authentication.
With a client associated to the rogue network, credentials may be
captured if the client's software attempts to automatically connect or
re-connect to a network service.  For example, rogue mail servers can
capture credentials from clients connecting to clear-text
mail services such as POP3 and IMAP.  

In some cases, the client's
credentials may be used against the client itself.  Older variants of
the Microsoft Networking NTLM authentication mechanism are vulnerable
to a man-in-the-middle attack where the attacker may proxy the
challenge-response protocol in order to authenticate as the client to
the server.
%If Windows networking services are enabled on the client
%machine, this attack may be performed by causing the client to connect
%through the attacker back to itself.

If a host is compromised, an agent may be placed on it that yields
remote control to the attacker.  The agent will attempt to establish
communication with the attacker whenever a network connection is
present.  This effectively gives the attacker access to all the
networks the wireless client has access to, exploiting the inherent
mobility of wireless clients.  This effectively makes the security of
every network the client connects to dependent upon the security of
all other networks they connect to.

% Why?
% stick firewalls deployed by default
% active attacks less likely
% client protocols are still chatty, send probes
% hostile replies are avenue for attack

%% The proliferation of firewalls with default deny policies enabled and
%% the overall reduction in attack surface areas has driven an attacker
%% to adjust attack procedures to compensate.  This has resulted in a
%% shift in focus from remote service-oriented attacks toward client-side
%% application attacks.
 
%The basic strategy with client-side attacks can be directed in a
%number of ways.  A direct approach similar to server-side attacks can
%be executed with limited modification of methodologies.

%% By abusing the inherent nature of client hosts initiating
%% communication requests with servers, in combination with the frequent
%% lack of stored host authentication information, a larger attack area
%% is exposed on client systems.  The most frequently used network
%% protocols are vulnerable to man in the middle (MitM) attacks.
%% Allthough more sensitive information is typically communicated over
%% secure channels, these methods are often subject to user interaction
%% to resolve ambiguities with certificate-based identification
%% anomalies.

%% In addition, The ability of an attacker to leverage wireless attacks
%% on the metropolitan scale has given rise to numerous new attack
%% vectors.  To leverage a wireless client host's mobile nature,
%% autonomous agents can be deployed with a phone-home style payload to
%% help coordinate internal attacks.

Our future work will further explore the vulnerabilities in
client-side functionality and client mobility and investigate the
construction of a client-side attack toolkit to demonstrate these
risks.

%Several weaknesses in WEP have been identified and addressed
%in recent times, one such example is the advent of weak IV avoidance
%by 802.11 hosts, thereby mitigating the effectiveness of some network
%based attacks.  Areas for exploitation still exist when considering
%client-side attack scenarios.

\bibliographystyle{ieeetr}
\bibliography{aawns}
\end{document}
